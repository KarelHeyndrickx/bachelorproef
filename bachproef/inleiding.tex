%%=============================================================================
%% Inleiding
%%=============================================================================

\chapter{Inleiding}
\label{ch:inleiding}

%De inleiding moet de lezer net genoeg informatie verschaffen om het onderwerp te begrijpen en in te zien waarom de onderzoeksvraag de moeite waard is om te onderzoeken. In de inleiding ga je literatuurverwijzingen beperken, zodat de tekst vlot leesbaar blijft. Je kan de inleiding verder onderverdelen in secties als dit de tekst verduidelijkt. Zaken die aan bod kunnen komen in de inleiding~\autocite{Pollefliet2011}:

Augmented reality ook wel AR genoemd, is een term waarvan technici altijd staan te popelen maar vaak fout begrepen en gedefinieerd wordt. Het is een soort interactieve, realiteit gebaseerde omgeving dat gebruik maakt van computer gegenereerde visualisatie, geluid en effecten om de gebruikers wereld uit te breiden. In tegenstelling tot Virtual Reality (VR) waar de gebruikers met behulp van een speciale isolerende bril in een volledig virtuele wereld terecht komen. Alhoewel het concept achter AR al decennia in de business wereld aanwezig is begint het nu nog maar zijn weg te maken naar de consumentenwereld. Sinds enkele jaren zijn er al een tal van applicaties of toepassingen ontstaan waarvan u mogelijks al in aanraking mee bent gekomen. Denk bijvoorbeeld aan de gezichtsfilters die u met de camera van Facebook messenger of snapchat kan uitproberen of aan Pokemon GO, waarbij spelers pokemons in de echte wereld kunnen visualiseren. Buiten deze 3 voorbeelden die dienen ter amusement zijn er ook een tal van andere use cases voor AR. Om even het nut en de opportuniteiten van AR te illustreren zijn er hieronder enkele opgesomd. 


\begin{itemize}
	\item \textbf{Productverkoop - IKEA Place:} IKEA Place is een app gemaakt door het Zweedse meubiliar bedrijf Ikea. Ikea heeft de opkomst van AR heel goed benut door als één van de eersten een prachtig en afgewerkte app te maken waarin u als gebruikers de meubels van Ikea in uw omgeving kunt visualiseren. De app bevat een bibliotheek van duizenden Ikea producten waarmee u kan testen of het tot uw kamer behoort of niet. Ikea heeft hier zodanig veel in geïnvesteerd dat ze er zelf voor gezorgd hebben dat indien u een meubelstuk met AR plaatst in uw kamer van een vijf tal centimeter hoog, het een geluid maakt alsof u echt dat meubelstuk zou laten vallen van 10 centimeter hoog. 
	\item \textbf{Dienstverkoop - Magicplan:} Magicplan is ook app die gemaakt is voor de doe-het-zelvers die een plattegrond willen bekomen van hun thuis of omgeving. In tegenstelling tot het pakken van een meetlint en elke hoek van de kamer af te gaan kunnen de doe-het-zelvers nu met magicplan simpelweg door de kamerl lopen en met hun camera hoekpunten zetten. De app berekent op zijn mysterieuze wijze hoe lang elke afstand is en toont wanneer u klaar bent een plattegrond die ook zichtbaar is in 3D. 
	\item \textbf{Productverbetering - Google maps AR:} Google heeft op 11 februari 2019 de Maps AR functionaliteit geïntroduceerd. Voorheen kon u, indien u bijvoorbeeld 5 straten verder wou geraken, kijken op de kaart waar u was en vervolgens stappen naar de locatie. Het enige probleem hierbij was dat het blauwe bolletje dat u voorstelde soms bewoog terwijl u stilstond. Dit is omdat de GPS localisatie niet altijd perfect werkt. Google had nu echter besloten dit probleem aan te pakken met behulp van AR en Machine Learning \footnote{Machine Learning: machinaal leren is een breed onderzoeksveld binnen Artificiële intelligentie. Hierbij wordt er gefocust op de ontwikkeling van methodes, algoritmen en technieken waarmee computers bepaalde functies aangeleerd kunnen worden.}. Met de app kan u nu gewoon rondkijken met uw camera waardoor Google de omgeving scant, herkent aan de hand van de gebouwen of omgeving en vervolgens de weg toont met pijltjes. 
	\item \textbf{Educatie - Human Anatomy Atlas 2018:} Deze app wordt gebruikt door gezondheidsprofessionals, lesgevers en medische studenten over de hele wereld om te kunnen kijken in het menselijk lichaam. U kan de met de AR functionaliteit een menselijk lichaam plaatsen op uw tafel en zo de anatomie bestuderen.
	\item \ldots

\end{itemize}

AR is dus zeker een onderwerp waar er heel veel mogelijkheden achter zitten. Maar ook moet er nog veel onderzoek naar gedaan worden. Want het probleem bij alle bovenstaande voorbeelden is dat u telkens er een app voor hoeft te downloaden. En indien u fan bent van verscheidene bedrijven kan u zo opeens 40 apps hebben, allemaal voor maar één specifieke functionaliteit. Dus wat als al de voorbeelden hierboven in één app konden? Een app waarmee uw accounts worden opgeslagen en u niet telkens van app hoeft te veranderen? De webbrowser! 

%\begin{itemize}
%  \item context, achtergrond
%  \item afbakenen van het onderwerp
%  \item verantwoording van het onderwerp, methodologie
%  \item probleemstelling
%  \item onderzoeksdoelstelling
%  \item onderzoeksvraag
%  \item \ldots
%\end{itemize}

\section{Probleemstelling}
\label{sec:probleemstelling}

In tegenstelling tot AR die nu al een aantal jaren bestaat voor apps, ligt AR op het web nog tamelijk wat achter. Bedrijven die bv duurdere producten, meubilaire producten of exclusieve producten verkopen op het web kunnen hier nog niet optimaal van profiteren. 

Dit komt omdat de web AR-technologieën nog in hun beginfase zitten. Zo heeft bijvoorbeeld Google hun web AR-technologie recent geleden beschikbaar gesteld aan ontwikkelaars maar zijn deze nog niet officieel, veilig en testbaar voor de gewone gebruiker. Ontwikkelaars kunnen dus wel al web AR-applicaties maken maar deze zijn nog niet zichtbaar voor de gewone gebruikers. 

In deze bachelorproef wordt er onderzoek gedaan naar hoe deze bedrijven alsnog aan de slag kunnen gaan met AR op het web en wat zij allemaal kunnen doen om hun producten te visualiseren op hun website met AR. Zodat wanneer Google een stabiele versie van Chrome aanbiedt die AR ondersteunt, gebruikers direct kunnen genieten van de reeds gemaakte AR-applicaties.

%Uit je probleemstelling moet duidelijk zijn dat je onderzoek een meerwaarde heeft voor een concrete doelgroep. De doelgroep moet goed gedefinieerd en afgelijnd zijn. Doelgroepen als ``bedrijven,'' ``KMO's,'' systeembeheerders, enz.~zijn nog te vaag. Als je een lijstje kan maken van de personen/organisaties die een meerwaarde zullen vinden in deze bachelorproef (dit is eigenlijk je steekproefkader), dan is dat een indicatie dat de doelgroep goed gedefinieerd is. Dit kan een enkel bedrijf zijn of zelfs één persoon (je co-promotor/opdrachtgever).

\section{Onderzoeksvraag}
\label{sec:onderzoeksvraag}

%Wees zo concreet mogelijk bij het formuleren van je onderzoeksvraag. Een onderzoeksvraag is trouwens iets waar nog niemand op dit moment een antwoord heeft (voor zover je kan nagaan). Het opzoeken van bestaande informatie (bv. ``welke tools bestaan er voor deze toepassing?'') is dus geen onderzoeksvraag. Je kan de onderzoeksvraag verder specifiëren in deelvragen. Bv.~als je onderzoek gaat over performantiemetingen, dan 

In deze bachelorproef is er niet juist één specifieke onderzoeksvraag. Er is eerder één onderzoeksvraag die zich verdeelt in verdere deelvragen. De onderzoeksvraag luidt :"Hoe kan ik AR implementeren op mijn website?" Deze vraag wakkert ook de vragen: Wat voor eigen ontworpen interacties kunnen de hierboven vermelde bedrijven met hun gegenereerde 3D visualisaties verkrijgen? Hoe kan dit mogelijk gemaakt worden voor zowel iOS als Android? Hoe kan zo een bedrijf er voor zorgen dat ze hun product voor zowel iOS als Android toestellen kunnen visualiseren? Op welke doelgroep zouden deze bedrijven zich moeten richten?


\section{Onderzoeksdoelstelling}
\label{sec:onderzoeksdoelstelling}

De doelstelling van deze bachelorproef is om een overzicht te verkrijgen welke paden een bedrijf kan nemen om AR functionaliteit te implementeren voor hun product. Welke is het meest rendabel en welke kan voor uw bedrijf goed zijn voor uw imago. 

\section{Opzet van deze bachelorproef}
\label{sec:opzet-bachelorproef}

% Het is gebruikelijk aan het einde van de inleiding een overzicht te
% geven van de opbouw van de rest van de tekst. Deze sectie bevat al een aanzet
% die je kan aanvullen/aanpassen in functie van je eigen tekst.

De rest van deze bachelorproef is als volgt opgebouwd:

In Hoofdstuk~\ref{ch:stand-van-zaken} wordt een overzicht gegeven van de stand van zaken binnen het onderzoeksdomein, op basis van een literatuurstudie.

In Hoofdstuk~\ref{ch:methodologie} wordt de methodologie toegelicht en worden de gebruikte onderzoekstechnieken besproken om een antwoord te kunnen formuleren op de onderzoeksvragen.

% TODO: Vul hier aan voor je eigen hoofstukken, één of twee zinnen per hoofdstuk

In Hoofdstuk~\ref{ch:conclusie}, tenslotte, wordt de conclusie gegeven en een antwoord geformuleerd op de onderzoeksvragen. Daarbij wordt ook een aanzet gegeven voor toekomstig onderzoek binnen dit domein.

