%==============================================================================
% Sjabloon onderzoeksvoorstel bachelorproef
%==============================================================================
% Gebaseerd op LaTeX-sjabloon ‘Stylish Article’ (zie voorstel.cls)
% Auteur: Jens Buysse, Bert Van Vreckem

\documentclass[fleqn,10pt]{voorstel}

%------------------------------------------------------------------------------
% Metadata over het voorstel
%------------------------------------------------------------------------------

\JournalInfo{HoGent Bedrijf en Organisatie}
\Archive{Bachelorproef 2018 - 2019} % Of: Onderzoekstechnieken

%---------- Titel & auteur ----------------------------------------------------

% TODO: geef werktitel van je eigen voorstel op
\PaperTitle{Proof of concept: het visualiseren van augmented reality in een browser}
\PaperType{Onderzoeksvoorstel Bachelorproef} % Type document

% TODO: vul je eigen naam in als auteur, geef ook je emailadres mee!
\Authors{Karel Heyndrickx\textsuperscript{1}} % Authors
\CoPromotor{Wouter Dewispelaere\textsuperscript{2} (Starring Jane)}
\affiliation{\textbf{Contact:}
	\textsuperscript{1} \href{mailto:karel.heyndrickx.y7711@student.hogent.be}{karel.heyndrickx.y7711@student.hogent.be};
	\textsuperscript{2} \href{mailto:wouter.dewispelaere@starringjane.com}{wouter.dewispelaere@starringjane.com};
}

%---------- Abstract ----------------------------------------------------------

\Abstract{Augmented reality ook wel AR genoemd, is een term waarvan technici altijd staan te popelen maar vaak fout begrepen en gedefinieerd wordt. Alhoewel het concept erachter al decennia in de business wereld aanwezig is begint het nu nog maar zijn weg te maken naar de consumentenwereld. 
Op dit moment is AR echter nog niet optimaal voor implementatie bij web gebaseerde use cases maar zitten er wel gigantische opportuniteiten achter. Zo zou bijvoorbeeld een kledingzaak met behulp van AR een klant een kledingstuk van uit de webshop rechtstreeks kunnen tonen in de kamer van die klant. Op die manier hoeft de klant het huis niet te verlaten, kan die eender welk artikel bekijken en valt de mogelijkheid weg dat het artikel niet meer beschikbaar is in de winkel.
Deze bachelorproef onderzoekt de visualisatiemogelijkheden marker based AR, markerless AR en location based AR, hoe implementeerbaar deze zijn en voert een proof of concept uit naar gelang de resultaten om te kijken of het rendabel kan zijn voor een bedrijf hiermee aan de slag te gaan. Er wordt verwacht dat niet alle technologieën even evident en haalbaar zijn maar dat er toch reeds één is waarmee AR web use cases kunnen voldaan worden. 
}

%---------- Onderzoeksdomein en sleutelwoorden --------------------------------
% TODO: Sleutelwoorden:
%
% Het eerste sleutelwoord beschrijft het onderzoeksdomein. Je kan kiezen uit
% deze lijst:
%
% - Mobiele applicatieontwikkeling
% - Webapplicatieontwikkeling
% - Applicatieontwikkeling (andere)
% - Systeembeheer
% - Netwerkbeheer
% - Mainframe
% - E-business
% - Databanken en big data
% - Machineleertechnieken en kunstmatige intelligentie
% - Andere (specifieer)
%
% De andere sleutelwoorden zijn vrij te kiezen

\Keywords{Onderzoeksdomein. Webapplicatieontwikkeling --- Augmented Reality --- 3D-Visualisatie} % Keywords
\newcommand{\keywordname}{Sleutelwoorden} % Defines the keywords heading name

%---------- Titel, inhoud -----------------------------------------------------

\begin{document}
	
	\flushbottom % Makes all text pages the same height
	\maketitle % Print the title and abstract box
	\tableofcontents % Print the contents section
	\thispagestyle{empty} % Removes page numbering from the first page
	
	%------------------------------------------------------------------------------
	% Hoofdtekst
	%------------------------------------------------------------------------------
	
	% De hoofdtekst van het voorstel zit in een apart bestand, zodat het makkelijk
	% kan opgenomen worden in de bijlagen van de bachelorproef zelf.
	%---------- Inleiding ---------------------------------------------------------

\section{Introductie} % The \section*{} command stops section numbering
\label{sec:introductie}

Augmented reality is een soort interactieve, realiteit gebaseerde omgeving dat gebruik maakt van computer gegenereerde visualisatie, geluid en effecten om de gebruikers wereld uit te breiden. In tegenstelling tot Virtual Reality (VR) waar de gebruikers in een volledig virtuele wereld terecht komen. 
Wanneer een gebruiker wil interageren met een gegenereerde wereld zijn er eerst nog enkele vragen die de AR applicatie moet beantwoorden. Wat moet er precies getoond worden in de gegenereerde wereld en waar moet het getoond worden? Hoe deze vragen beantwoord worden hangt af van het type van de AR applicatie. Zoals vermeld in  ~\textcite{Paladini2018} zijn er hier 3 types in te onderscheiden. In sommige gevallen moet de applicatie weten naar wat hij kijkt. Dit wordt marker based AR genoemd. In een andere situatie is dit echter niet nodig en wordt de ruimte rond de gebruiker gescand om te wereld te kunnen herkennen. Hierbij wordt er gesproken over markerless AR. Bij het derde type moet de applicatie de gebruiker zijn locatie weten. Dit type noemt men location based AR.  
Vandaag de dag wordt er meer en meer gebruik gemaakt van browsers om alle dagelijkse taken te vervullen. Op die manier heeft de gebruiker maar 1 tool nodig in plaats van verscheidene tools waaronder apps, browsers en computer programma’s en wordt er dus een uniformiteit gecreëerd. Aangezien er in praktijk van de 3 bovenstaande types AR applicaties al een tal van downloadbare apps aanwezig zijn maar echter nauwelijks webapplicaties is er dus noodzaak aan een onderzoek voor het implementeren van AR applicaties in de browser. Het doel van deze proof of concept is om een realistisch beeld te verkrijgen zowel over de haalbaarheid van het implementeren van AR in een browser alsook de vraag of het voor sommige bedrijven voordeel biedt hierin te investeren. 



%---------- Stand van zaken ---------------------------------------------------

\section{State-of-the-art}
\label{sec:state-of-the-art}

Op het moment van schrijven van deze bachelorproef zijn er door anderen reeds projecten gedaan omtrent de AR technologieën. Zo staan er online enkele handleidingen hoe het mogelijk is marker based AR te implementeren in een website. Een voorbeeld hiervan is te volgen in het artikel ~\textcite{Etienne2017}. Hier wordt er uitgelegd hoe een ontwikkelaar aan zijn eigen website een marker based AR functie kan toevoegen, hiervan het model dat getoond wordt op het merkteken kan veranderen en het merkteken waarop het model staat kan veranderen. Over markerless AR is er daarentegen veel minder informatie omdat deze technologie nog in de alfa fase zit van ontwikkeling. Echter is er wel reeds een downloadbaar project rond gemaakt door Google en is dit raadpleegbaar via de artikels ~\textcite{Stanush2018} en ~\textcite{Ali2018}. Dit project is een voorbeeld van een artikelpagina zoals Wikipedia waarin er een astronaut wordt beschreven en weergegeven via AR. Er wordt hierbij gebruik gemaakt van smart terrain detection om de ruimte rond de gebruiker te identificeren en zo de astronaut in die ruimte te plaatsen. 



% Voor literatuurverwijzingen zijn er twee belangrijke commando's:
% \autocite{KEY} => (Auteur, jaartal) Gebruik dit als de naam van de auteur
%   geen onderdeel is van de zin.
% \textcite{KEY} => Auteur (jaartal)  Gebruik dit als de auteursnaam wel een
%   functie heeft in de zin (bv. ``Uit onderzoek door Doll & Hill (1954) bleek
%   ...'')


%---------- Methodologie ------------------------------------------------------
\section{Methodologie}
\label{sec:methodologie}

In deze bachelorproef zal er eerst onderzoek gedaan worden naar de huidige implementatiemogelijkheden van de AR technologieën en zal er vervolgens één of meerdere webapplicaties gebouwd worden om deze onder de loep te nemen. Het opbouwen van de webapplicaties zal gedaan worden met behulp van online tutorials en relevante informatie omtrent deze technologieën. Voorbeelden hiervan zijn ~\textcite{Medley2018} en ~\textcite{Etienne2017}.

%---------- Verwachte resultaten ----------------------------------------------
\section{Verwachte resultaten}
\label{sec:verwachte_resultaten}

Aangezien er in deze bachelorproef onderzoek gedaan wordt naar de drie AR technologieën zijn er drie verwachte resultaten. Marker based AR lijkt hierbij de eenvoudigste van de drie om te gebruiken in een webapplicatie aangezien er hierover het meest informatie te vinden is en er reeds verscheidene tutorials zijn. Daarnaast lijkt Markerless based AR hier het meest intrigerende te zijn aangezien deze technologie nog in de alfa fase zit. Toch moet het mogelijk zijn om ook dit te implementeren in een webapplicatie. Vervolgens is er nog location based AR. Aangezien er rond deze technologie nauwelijks informatie te vinden is lijkt het op dit moment nog onmogelijk om dit te gebruiken in een webapplicatie. 

%---------- Verwachte conclusies ----------------------------------------------
\section{Verwachte conclusies}
\label{sec:verwachte_conclusies}

Het uitvoeren van dit onderzoek en het opstellen van een proof of concept zou moeten aantonen dat met marker based AR en markerless AR het zeker rendabel is voor een bedrijf hiermee aan de slag te gaan. Aangezien er met marker based AR al tal van opportuniteiten zijn en deze niet veel werk is om te implementeren zou deze de meest aangeraden technologie moeten zijn. 


	
	%------------------------------------------------------------------------------
	% Referentielijst
	%------------------------------------------------------------------------------
	% TODO: de gerefereerde werken moeten in BibTeX-bestand ``voorstel.bib''
	% voorkomen. Gebruik JabRef om je bibliografie bij te houden en vergeet niet
	% om compatibiliteit met Biber/BibLaTeX aan te zetten (File > Switch to
	% BibLaTeX mode)
	
	\phantomsection
	\printbibliography[heading=bibintoc]
	
\end{document}
